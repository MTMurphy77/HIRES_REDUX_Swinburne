\documentclass[11pt,letterpaper,dvips]{article}
 
 
\usepackage{latexsym}
\usepackage{fancybox}
\usepackage{graphicx}
\usepackage{amssymb}
\usepackage{color}
%\usepackage{amstex}
\usepackage{ulem}
\usepackage{float}
 
\pretolerance=10000
\textwidth=7.0in
\textheight=9.7in
\voffset = -0.3in
\topmargin=0.0in
\headheight=0.00in
\hoffset = -0.3in
\headsep=0.00in
\oddsidemargin=0in
\evensidemargin=0in
\parindent=2em
\parskip=1.5ex
 
\def\rexp{{\rm e}}
\def\katzt{\ket{\alpha, t_0=0; t}}
\def\katt{\ket{\alpha, t_0; t}}
\def\usc{\mathcal{U}}
\def\tsc{\mathcal{T}}
\special{papersize=8.5in,11in}
\newcommand{\nid}{\noindent}
\newcommand{\ew}{W_\lambda}
\def\aap{A \& A}
\def\aj{AJ}
\def\apj{ApJ}
\def\apjl{ApJL}
\def\apjs{ApJS}
\def\apjsupp{ApJS}
\def\araa{ARAA}
\def\mnras{MNRAS}
\def\nat{Nature}
\def\pasp{PASP}
\def\hangpara{\par\hangindent 25pt\noindent}
\def\Lya{Ly$\alpha$}
\def\lya{Ly$\alpha$}
\def\Lyb{Ly$\beta$ }
\def\lyb{Ly$\beta$ }
\def\Lyg{Ly$\gamma$ }
\def\Lyd{Ly$\delta$ }
\def\etal{et al. }
\def\kms{km~s$^{-1}$ }
\def\mkms{{\rm \; km\,s\,^{-1}}}
\def\bmkms{{\rm\bf \; km\,s\,^{-1}}}
\def\L{$\lambda$}
\def\s-1{s$^{-1}$}
\def\Hz-1{Hz$^{-1}$}
\def\pohf{\rm p_\ohf}
\def\prhf{\rm p_\rhf}
\def\ohf{\frac{1}{2}}
\def\nohf{\frac{-1}{2}}
\def\rhf{\frac{3}{2}}
\def\sohf{\rm s_\ohf}
\def\esp{\Delta E_{2s, 2p}}
\def\chal{\frac{\Delta \alpha}{\alpha}}
\def\ylab{y_{\rm lab}}
\def\yaY{y \; {\rm and} \; Y}
\def\dadz{\frac{1}{\alpha} |\frac{{\rm d} \alpha}{{\rm d} z}|}
\def\abdadz{\frac{1}{\alpha} \left|\frac{{\rm d} \alpha}{{\rm d} z}\right|}
\def\abdadt{\frac{1}{\alpha} \left|\frac{{\rm d} \alpha}{{\rm d} t}\right|}
\def\sci#1{{\; \times \; 10^{#1}}}
\def\yold{y_{\rm old}}
\def\und#1{{\rm \underline{#1}}}
\def\dlmb{\Delta \lambda}
\def\abchal{| \chal |}
\def\lbar{\bar\lambda}
\def\elc{\rm e^-}
\def\qnn#1{\eqno {\rm {#1}}}
\def\l1l2{\lambda_2 \; {\rm and} \; \lambda_1}
\def\id{\indent}
\def\wid{\id\id}
\def\rid{\id\id\id}
\def\uid{\id\id\id\id}
\def\imp{\, \Rightarrow \>}
\def\di{\partial}
\def\ba{{\bf a}}
\def\bx{{\bf x}}
\def\bv{{\bf v}}
\def\bu{{\bf u}}
\def\bq{{\bf q}}
\def\br{{\bf r}}
\def\bs{{\bf s}}
\def\bp{{\bf p}}
\def\bj{{\bf j}}
\def\bk{{\bf k}}
\def\bm{{\bf m}}
\def\bn{{\bf n}}
\def\bd{{\bf d}}
\def\bg{{\bf g}}
\def\by{{\bf y}}
\def\bB{{\bf B}}
\def\bE{{\bf E}}
\def\bD{{\bf D}}
\def\bH{{\bf H}}
\def\bA{{\bf A}}
\def\bI{{\bf I}}
\def\bJ{{\bf J}}
\def\bK{{\bf K}}
\def\bL{{\bf L}}
\def\bS{{\bf S}}
\def\bF{{\bf F}}
\def\bN{{\bf N}}
\def\bP{{\bf P}}
\def\bR{{\bf R}}
\def\bU{{\bf U}}
\def\bM{{\bf M}}
\def\bY{{\bf Y}}
\def\bOm{{\bf \Omega}}
\def\bom{{\bf \omega}}
\def\bell{{\bf \ell}}
\def\di{\partial}
\def\ddi#1#2{\frac{\di #1}{\di #2}}
\def\gmm{\sqrt{1 - \frac{v^2}{c^2}}}
\def\eng{\varepsilon}
\def\dd#1#2{\frac{d #1}{d #2}}
\def\dt#1{\frac{d #1}{dt}}
\def\ddt{\frac{d}{d t}}
\def\ddit{\frac{\di}{\di t}}
\def\arr{\to \quad}
\def\smm{\sum\limits}
\def\intl{\int\limits}
\def\ointl{\oint\limits}
\def\liml{\lim\limits}
\def\prodl{\prod\limits}
\def\grd{{\bf \nabla}}
\def\crl{{\bf \grd} \times}
\def\dv{{\bf \grd} \cdot}
\def\ec{\frac{e}{c} \,}
\def\oc{\frac{1}{c}}
\def\xht{{\; \rm \hat x}}
\def\yht{{\; \rm \hat y}}
\def\zht{{\; \rm \hat z}}
\def\rht{{\; \rm \hat r}}
\def\pht{{\; \rm \hat \phi}}
\def\iht{{\; \rm \hat i}}
\def\jht{{\; \rm \hat j}}
\def\kht{{\; \rm \hat k}}
\def\theht{{\; \rm \hat \theta}}
\def\nht{{\; \rm \hat n}}
\def\oht{{\; \rm \hat 1}}
\def\tht{{\; \rm \hat 2}}
\def\vht{{\; \rm \hat v}}
\def\thht{{\; \rm \hat 3}}
\def\Rht{{\; \rm \hat R}}
\def\cyc{\Omega_c}
\def\dl{{d {\bf \ell}}}
\def\da{{\rm d {\bf a}}}
\def\ds{{\rm d {\bf s}}}
\def\smt{{\smll \rm T}}
\def\opi#1{\frac{1}{#1 \pi}}
\def\strs{\underline {\underline {\rm T}}}
\def\dtr{\, d^3 r}
\def\dftr{d^3 r \,}
\def\dtp{\, d^3 p}
\def\dftp{d^3 p \,}
\def\abrr{|\br - \br'|}
\def\rr{(\br - \br')}
\def\orr#1{\frac{#1}{\abrr}}
\def\trr#1{\frac{#1}{\abrr^3}}
\def\drr{\delta \rr}
\def\Grn{G(\br, \br')}
\def\Ylm{Y_{\ell m}}
\def\snth{\sin\theta}
\def\csth{\cos\theta}
\def\invr#1{\frac{1}{#1}}
\def\fpc{\frac{4 \pi}{c}}
\def\lgnl{P_\ell (\csth)}
\def\lgnn#1{{P_{#1} (\csth)}}
\def\thph{\theta, \phi}
\def\eht1{{\rm \hat e_1}}
\def\ehtb{{\rm \hat e_2}}
\def\iot{i \omega t}
\def\kr{\bk \cdot \br}
\def\ele{{\rm e^-}}
\def\iwc{\frac{i \omega}{c}}
\def\iwt{i \omega t}
\def\plr{{\rm \hat \epsilon}}
\def\skp{\smallskip}
\def\mkp{\medskip}
\def\bkp{\bigskip}
\def\elc{\rm e^{-s}}
\def\bmx{\left[\matrix}
\def\Det{{\rm Det \,}}
\def\bra#1{{<\negmedspace#1|}}
\def\ket#1{{|#1\negmedspace>}}
\def\kalp{\ket{\alpha}}
\def\kbet{\ket{\beta}}
\def\balp{\bra{\alpha}}
\def\bbet{\bra{\beta}}
\def\inpr#1#2
\def\otpr#1#2{{|#1><#2|}}
\def\dgg{\dagger}
\def\dual{\Leftrightarrow}
\def\hbm{\frac{\hbar^2}{2 m}}
\def\hbu{\frac{\hbar^2}{2 \mu}}
\def\mbh{\frac{2 m}{\hbar^2}}
\def\ih{\frac{i}{\hbar}}
\def\kjm{\ket{j m}}
\def\bjm{\bra{j m}}
\def\bjmp{\bra{j' m'}}
\def\ylm{Y_{\ell m}}
\def\pof#1{(\frac{#1}{2})}
\def\tqk{T_q^k}
\def\ptm{\frac{p^2}{2 m}}
\def\psn{\psi_n}
\def\tph{\frac{2 \pi}{\hbar}}
\def\plr{{\hat \epsilon}}
\def\Veff{V_{\rm eff}}
\def\rhs{\rho^{(s)}}
\def\rhis{\rho_i^{(s)}}
\def\tilH{\tilde H}
\def\tpi{2 \pi i}
\def\prin{\intl_P}
\def\ltk{\left [ \,}
\def\ltp{\left ( \,}
\def\ltb{\left \{ \,}
\def\rtk{\, \right  ] }
\def\rtp{\, \right  ) }
\def\rtb{\, \right \} }
\def\dc{\tilde d}
\def\bskp{\bigskip}
\def\nV{\Omega}
\def\d3x{d^3 x}
\def\mtI{\b{\b I}}
\def\cmma{\;\;\; ,}
\def\perd{\;\;\; .}
\def\delv{\Delta v}
\def\eikr{{\rm e}^{i \bk \cdot \br}}
\def\rmikr{{\rm e}^{i \vec k \cdot \vec r}}
\def\rmnikr{{\rm e}^{-i \vec k \cdot \vec r}}
\def\eikx{{\rm e}^{i \bk \cdot \bx}}
\def\muo{\mu_0}
\def\epo{\epsilon_0}
\def\rme{{\rm e}}
\def\dnu{\Delta \nu_D}
\def\rAA{\AA \enskip}
\def\halp{{\rm H_\alpha}}
\def\hbet{{\rm H_\beta}}
\def\N#1{{N({\rm #1})}}
\def\ekz{\rme^{i ( k_z z - \omega t )}}
\def\cm#1{\, {\rm cm^{#1}}}
\def\bcm#1{\; {\rm \mathbf{cm^{#1}}}}
\def\rmK{\; \rm K}
\def\rmm#1{\; {\rm #1}}
\def\Nperp{N_\perp}
\def\spkr#1{{\big \bf {#1}}}
\def\inA{\quad $\bullet$ \quad}
\def\inB{\quad\quad $\diamond$ \quad}
\def\inC{\quad\quad\quad $\circ$ \quad}
\def\sigfc{\sigma_{f_c}}
\def\sech{\rm sech}
\def\blck{\color[named]{Black}}
\def\blue{\color[named]{Blue}}
\def\skyblue{\color[named]{SkyBlue}}
\def\red{\color[named]{Red}}
\def\black{\color[named]{Black}}
\def\green{\color[named]{Green}}
\def\white{\color[named]{White}}
\def\yellow{\color[named]{Yellow}}
\def\purple{\color[named]{Purple}}
\def\maroon{\color[named]{Maroon}}
\def\redv{\color[named]{RedViolet}}
\def\mahog{\color[named]{Mahogany}}
\def\prb{$\frac{\delta \rho}{\rho}$}
\def\trd{\frac{2}{3}}
\def\mdn{\bar \rho}
\def\ss{v_s^2}
\def\msol{M_\odot}
\def\W#1{{W({\rm #1})}}
\def\Nav{N_a (v)}
\def\f#1{{f_{\rm #1}}}
\def\opt{$d \tau / d z \; $}
\def\btau{\bar\tau (v_{pk}) / \sigma (\bar \tau)}
\def\Ipk{I(v_{pk})/I_{c}}
\def\vrot{v_{rot}}
\def\sphr{\sqrt{R^2 + Z^2}}
\def\mbf{\mathbf}
\def\mgsqa{\, {\rm mag} / \Box ''}
\newcommand{\ialii}{Al{\sc II}}
\newcommand{\iari}{Ar{\sc I}}
\newcommand{\ini}{N{\sc I}}
\newcommand{\ioi}{O{\sc I}}
\newcommand{\iovi}{O{\sc VI}}
\newcommand{\iovii}{O{\sc VII}}
\newcommand{\ioviii}{O{\sc VIII}}
\newcommand{\ifeii}{Fe{\sc II}}
\newcommand{\ifeiii}{Fe{\sc III}}
\newcommand{\ispii}{S{\sc II}}
\newcommand{\iznii}{Zn{\sc II}}
\newcommand{\icaii}{Ca{\sc II}}
\newcommand{\inai}{Na{\sc I}}
\newcommand{\icii}{C{\sc II}}
\newcommand{\iciv}{C{\sc IV}}
\newcommand{\ihii}{H{\sc II}}
\newcommand{\ihi}{H{\sc I}}
\newcommand{\isiii}{Si{\sc II}}
\newcommand{\isiiv}{Si{\sc IV}}
\newcommand{\ici}{C{\sc I}}
\newcommand{\dcost}{d \negmedspace \cos\theta}
\newcommand{\mnvhi}{n_{\rm H {\sc I}}}
\newcommand{\nvhi}{$n_{\rm H {\sc I}}$}
\newcommand{\rem}{{\rm e}}
\def\otpr#1#2{{|#1><#2|}}
\def\avgq#1{<\negmedspace#1\negmedspace>}
\newcommand{\negm}{\negmedspace}
\newcommand{\dfhf}{^2\negm D_{\frac{5}{2}}}
\newcommand{\negt}{\negthinspace}
\newcommand{\nalph}{N/$\alpha$}
\newcommand{\alphh}{$\alpha$/H}
\newcommand{\nhi}{$N_{\rm HI}$}
\newcommand{\mnhi}{N_{\rm HI}}

\renewcommand{\descriptionlabel}[1]%
    {\hspace{\labelsep}\textsf{#1}}

\newenvironment{Aitemize}{%
 \renewcommand{\labelitemi}{$\bullet \;$}%
	\begin{itemize}}{\end{itemize}}

\newenvironment{Renumerate}{%
 \renewcommand{\theenumi}{\Roman{enumi}}
 \renewcommand{\labelenumi}{\theenumi)}
	\begin{enumerate}}{\end{enumerate}}

\newenvironment{Aenumerate}{%
 \renewcommand{\theenumi}{\Alph{enumi}}
 \renewcommand{\labelenumi}{\theenumi.}
	\begin{enumerate}}{\end{enumerate}}

\newenvironment{dmnd}{%
 \renewcommand{\labelitemi}{$\diamond \quad$}%
	\begin{itemize}}{\end{itemize}}

\newenvironment{Penumerate}{%
 \renewcommand{\labelenumi}{(\theenumi)}
	\begin{enumerate}}{\end{enumerate}}


 
\special{papersize=8.5in,11in}

\renewcommand{\theenumi}{\Roman{enumi}}

\begin{document}
\pagestyle{empty}
 

\center{{\Huge \bf IDL Image Reduction}}

\begin{enumerate}

{\Large \item Introduction: }
Should write a brief intro here.


{\Large  \item Requirements:  }
	\begin{itemize}
		\item $>$300M RAM (this can be modified)
                \item the XIDL package
		\item the IDL coyote package 
		\item the Goddard IDL package
		\item the Schlegel IDL package (compiled)
	\end{itemize}

\clearpage

{\Large  \item Setup }

  \begin{Aenumerate}

    \item Create a Raw/ directory and put all the raw data in it.  Move
	to the directory above Raw

    \item {\bf xdimg\_strct} :: This creates the IDL structure which 
	drives all of the subsequent programs.  It also creates a fits file
	of the structure (``struct.fits'') and a summary in the ASCII file 
	``image.list''.  After running the program, one should 
	modify the resulting structure as needed.  A brief summary of
	the structure is given below.  An example of this step
	is xdimg\_setup.pro \\
         \quad Example: {\bf xdimg\_strct}, n1\_strct, 'LRISR', 'Keck', /MKDIR\\
         \quad Time   : $<5$s per image

    \item {\bf distruct}

	{\small
	\begin{tabular}{lcl}
	  \hline
	  Tag & Type & Comment \\
	  \hline
          flg\_anly & 0 &      Analysis flag 0=Don't Analyse \\
          frame & 0L &         FRAME Number \\
          naxis1 &  0L &       NAXIS1 is helpful for determining sub-chip readouts, esp for NIRC2 \\
          naxis2 & 0L       &  NAXIS2 \\
          exp & 0.d &          Exposure time \\
          filter & ' ' &       Filter : U,B,V,R,I,H,Kp \\
          type & ' ' &         ObjTyp : OBJ, STD, TWI, DRK, ZRO, DFT, SPS \\
          Obj & ' ' &          Object Name \\
          Equinox & 0. &       EQUINOX \\
          Date & 0.0d &        Date of Obs \\
          UT & ' ' &           UT \\
          RA & ' ' &           RA \\
          DEC & ' ' &          DEC \\
          TEL & ' ' &          Telescope \\
          AM &   0. &          Airmass \\
          RMS &   0. &         RMS \\
          extinct &   0. &     Extinction (e.g. cirrus) \\
          CCD & ' ' &          CCD : LRISR, NIRC2, SITe3, etc.\\
          namp & 0 &           Namps \\
          coadds & 0L &        Number of coadds (For IR images) \\
          sampmode & 0L &      Sampmode \\
          multisam & 0L &      Multisam \\
          statsec & lonarr(4) &    Stat section \\
          gain & 0. &          Gain \\
          readno & 0. &        Read Noise \\
          satur  & 0. &        Saturation \\
          rootpth & ' ' &      Path of the Root \\
          img\_root & ' ' &    Root name (usually in Raw directory) \\
          med\_raw & 0L &      Median of the Raw image \\
          min\_raw & 0L &      Min of the Raw image \\
          max\_raw & 0L &      Max of the Raw image \\
          flg\_ov &  0L &       OV FILE?  0=No, 1=Yes   \\
          img\_ov &  ' ' &     Name of OV file (with directory) \\
          med\_ov &  0. &      Median of OV file \\
          flg\_drk & 0L &        Does necessary dark file exist? 0=NO, 1=Yes \\
          img\_drk & ' ' &     Name of dark file which should be used with this image. (with directory) \\
          flg\_drksub & 0L &    Has image been dark subtracted? 0=NO, 1=Yes \\
          img\_drksub & ' ' &  Name of the dark subtracted (and coadd divided) file (with directory) \\
          med\_drksub & 0. &   Median of dark subtracted, coadded divided file \\
          flg\_msk &  0L &      Mask FILE?  0=No, 1=Yes   \\
          img\_msk &  ' ' &    Name of Mask file \\
          flg\_skymsk &   0L &  Sky Mask FILE?  0=No, 1=Yes   \\
          img\_skymsk &  ' ' & Name of Sky Mask file \\
          flg\_final & 0L &     Final File? 0=No \\
          img\_final & ' ' &  Name of Final img \\
	  \hline
	\end{tabular}
	}

  \end{Aenumerate}

  \clearpage

  {\Large  \item CALIBRATIONS :: } \\
  The calibration section is split into two parts, the
  first for creating calibration files for use with optical images and
  the second for creating calibration files for use with infrared
  images.
  
  Note that at any point during the reductions you can exit out of
  IDL, and re-start the calibrations later by reading in the structure
  as follows: n1\_strct=mrdfits('struct.fits',1).  The structure is
  re-saved at each step in the process.


  {\Large OPTICAL CALIBRATIONS ::} \\
  Create Biases and Flats, including a Super-sky flat.
  An example of these steps is given in Notes/pro/xdimg\_calibs\_opt.pro.  You will
  need one set of flats but otherwise these steps are optional.
  For the flats, a non-linear correction is applied where applicable
  (e.g.\ SITe3)
  
  \begin{Aenumerate}
    
  \item {\bf xdimg\_bias} :: Create the Bias frame.  This routine ov
    subtracts the bias images and median combines.  It then deletes
    the ov images unless instructed otherwise.  Currently, the 
    'Tek5', 'Site\#1', 'LRISR' chips are supported.  Investigate the
    bias as you like (e.g.\ xatv, 'Bias/Bias.fits') \\
    \quad Example: xdimg\_bias, n1\_strct \\
    \quad Time   : 5 min
    
  \item {\bf xdimg\_dflat} :: Create the Dome Flats.  This routine ov
    subtracts the dome flat images and median combines.  It then deletes
    the ov images unless instructed otherwise.  It creates one normalized
    flat per filter.  Investigate the
    flats as you like (e.g.\ xatv, 'Flats/DFlatNR.fits'). \\
    \quad Example: xdimg\_dflat, n1\_strct \\
    \quad Time   : 10 min
    
  \item {\bf xdimg\_twiflt} :: Create the Twilight Flats.  This routine ov
    subtracts the twilight flat images and median combines.  It then deletes
    the ov images unless instructed otherwise.  It creates one normalized
    flat per filter.  Investigate the
    flats as you like (e.g.\ xatv, 'Flats/TFlatNR.fits'). \\
    \quad Example: xdimg\_twiflt, n1\_strct \\
    \quad Time   : 10 min
  \end{Aenumerate}
  
  {\Large INFRARED CALIBRATIONS ::} \\
  Create darks, flats, and a bad pixel mask.  An example of these
  steps is given in Notes/pro/xdimg\_calibs\_ir.pro.  You will need one set of
  flats for each filter, and a super-dark frame for each exposure time
  of science data.  The bad pixel mask is helpful but not necessary.

  \begin{Aenumerate}
    
  \item{\bf xdimg\_badpix} :: Create a bad pixel mask.  This routine
    uses the dark frames to pick out hot pixels and it uses
    well-exposed flat fields to pick out dead pixels.  Therefore the
    structure must include at least one of these types of frames.  You
    can set the DEADPERCENT and HOTPERCENT keywords to change the
    threshold of what is considered a bad pixel.  If multiple flats
    and darks are present the program goes through each frame looking
    for dead and hot pixels.  In this way it can locate ``sometimes''
    dead and hot
    pixels.  If running this program on NIRC2 data, you must run it once for the narrow camera and once for the wide camera since these are treated as separate instruments. \\
    \quad Example: xdimg\_badpix, n1\_struct, CCD='NIRC2W' \\
    
  \item{\bf xdimg\_dark} :: Create a super-dark frame for every type of dark frame that we have.  For example, there will be one super-dark frame for each unique combination of exposure time, coadds, sampmode, and nreads.  This routine median combines all the dark exposures of the same type.  Investigate the super-darks as you like (e.g. \ xatv,
    'Darks/Dark\_w10.000\_6\_3\_8.fits') The darks are saved with the NIRC2 camera name, the exposure
    time, the number of coadds, the sampmode, and the number of reads (in that order). \\
    \quad Example: xdimg\_dark, n1\_struct \\

  \item{\bf xdimg\_dflat} :: Create the dome flat.  When used with the
    /IR flag, this routine dark subtracts the flats with the
    super-dark frame of the same type, and then median combines,
    scaling by the median value in a small portion of the image.  It
    then deletes the dark subtracted images unless instructed
    otherwise.  It creates one normalized flat per filter, and per camera, of course.
    Investigate the flats as you like (e.g. \ xatv, 'Flats/DFlat\_w\_NKp.fits').\\
    \quad Example xdimg\_dflat, n1\_strct, /IR, CCD='NIRC2W', BPM='PixMask/BadPixMask\_w.fits' \\
    
  \end{Aenumerate}
  
  {\Large  \item Sky Flat :: } \\
  This is the preferred flat for analysis.  Although strictly a
  calibration file, this step can be more interactive than the previous
  and deals with the data images.  Ideally one has $>5$ well dithered
  images per filter.  The routine:

  \begin{Aenumerate}
    \item OV subtracts the object images (usually interactively with the /INTER 
	keyword).  If the /IR flag is supplied then the routine dark subtracts instead of OV subtracting.
    \item Runs a gui ({\bf xmkmask}) designed to make sky masks (suppressed
	with the /NOMSK keyword). 
    \item It then median combines with clipping. 
  \end{Aenumerate}
    	The ov images are generally kept. The routine creates one normalized 
	flat per filter.  Investigate the flats as you like 
	(e.g.\ xatv, 'Flats/SkyFltNR.fits'  OR \ xatv, 'Flats/SkyFlt\_w\_Nkp.fits').  \\
	Note: If xdimg\_skyflt fails due to problems with dynamic linking, it is possible that your installation of XIDL was faulty.  Update the XIDL CVS repository, and be sure to type bin/evilmake clean followed by bin/evilmake all while in XIDL.\\
        \quad OPT Example: xdimg\_skyflt, n1\_strct, /INTER \\
        \quad IR Example: xdimg\_skyflt, n1\_strct, /IR, CCD='NIRC2W', /NOMSK \\
	\quad Time   : 45min (masks), 10min combining

	\begin{itemize}
	   \item {\bf xmkmask} :: IDL gui for creating masks of a set of
	images.  Use the keyword /SKYMSK to create a skymask.  User
	can create rectangular, circular, and line masks.
		\begin{itemize}
		  \item Rectangle: LMB, LMB on corners of rectangle
		  \item Circular: LMB, LMB on center then outer radius of circle
		  \item Line: LMB, LMB, LMB on center of line, then one edge
			of the line and then the other center
		\end{itemize}
	After saving the first Mask, one can dither the same Mask on the next
	image using the dx, dy values [Note: minus signs require you type a 
	non-zero number first; this is confusing].  
	Set them one by one and hit return
	to update the image+mask.  Iterate.  Click on SVDITH when done or on
	NODITH to delete the current Mask.  You can then add new sub-masks 
	to the
	current image as desired.  The Masks are written to the Masks/
	directory by default and gzipped to save memory.
	\end{itemize}
	

{\Large  \item PHOTOMETRIC CALIBRATION} :: \\

	\begin{enumerate}

	  \item See xdimg\_stds.pro for an example.
	  \item {\bf xdimg\_proc } :: Process the standard field
            images.  Identify all STD images with the {\bf where}
            command, then run {\bf xdimg\_proc}.  You will need to
            choose which image to flatten with and input the root name
            (e.g.\ 'Flats/TSkyFltN' or 'Flats/DFlatN').  Notice you do
            not use the entire flat name, just the filterless portion.
            If dealing with infrared data, you will need to use the
            /IR flag, and you should provide the name of a bad pixel
            mask if you have one.  The resulting images are in the
            Final/ directory with an 'f\_' prefix.  Note that this
            procedure does NOT multiply by the gain by default.  Make
            sure the date is properly set for each image or is set globally! \\
            \quad OPT Example: STDS = where(n1\_struct.type EQ 'STD' AND
            n1\_struct.flg\_anly NE 0 ) \\
            \quad\quad\quad xdimg\_proc, n1\_struct, STDS, 'flatroot' \\
            \quad IR Example: STDS = where(n1\_struct.type EQ 'STD' AND n1\_struct.flg\_anly NE 0 and n1\_struct.ccd eq 'NIRC2W') \\
            \quad\quad\quad xdimg\_proc, n1\_struct, STDS, 'flatroot', /IR, BPM='PixMask/BadPixMask\_w.fits' \\
            \quad Time : Approx 15 sec / image
	        
	  \item {\bf xdimg\_starid} :: Identify the stars in each
            standard field.  This routine displays the image, allows
            one to flip it, directs the reader to read in a star list
            relevant to the field, and then facilitates the
            identification of all stars in that field.  It is not
            necessary to have the standard fields in the local List
            directory before starting this procedure.  (By default,
            the program will search the /IMG/Photometry/Lists
            directory.)  But it is advised to use the LST\_PATH
            keyword and copy over the list locally.  You'll want to be
            familiar with the order of the stardards in the list,
            because you'll have to click on the stars in the image one
            by one, in the order that they appear in the starlist.
            Once you have identified and centered the stars, you can
            save the list which is convenient given the poor quality
            of the Landolt coordinates and proper motions.  This
            program drives the program {\bf x\_starid} which is an IDL
            gui.  See its help page for more details.  Play around with the inorient keyword to get the standard stars to line up correctly on the image.  After running xdimg\_starid, there should be a new file in the Photo directory called std\_****.dat, and this file should contain the pixel positions of the standard stars.\\
            \quad Example: {\bf xdimg\_starid}, n1\_struct, inorient=4, LST\_PATH='/Users/pollack/foo/Lists/' \\
            \quad Time : Mostly human labor ($<$1 min / image)

	  \item {\bf xdimg\_stdmag} :: Calculate magnitudes for the stars.  This 
		routine takes the star positions from the previous step, 
		reads in each image and calculates $7''$ aperture magnitudes.
		The output is written to Photo/ by default (mag\_*).  Note: If you found that some of the standard stars are staruated, or fall on a bad column, or whatever, you may want to edit the ``mag'' files in the Photo directory before continuing on to the next step of photcal.  You'll have a chance in xdimg\_photcal to remove bad data points, but it may not be obvious which ones are bad. \\
         	\quad Example: {\bf xdimg\_stdmag}, n1\_struct \\
	        \quad Time   : Fast, 5s/image

	  \item {\bf xdimg\_photcal} :: Derive the final solution.  This routine
		takes the magnitudes from the previous step, reads in the
		Landolt info, and then creates and plots up the solution.  
		One can set an AM
		term instead of fitting for it with the SETAM keyword.  (For a good discussion on airmass, and to find values of AM to use in case you can't fit for it yourself, see Prochaska et al. 2001, AJ, 123, 2206.)
		Also the default is to only include Landolt stars with at least
		4 measurements.  This can be changed with the MIN\_NOBS and
		MIN\_MOBS keywords.  The output is written as 'soln\_*' in
		the Photo directory.  \\
         	\quad Example: {\bf xdimg\_photcal}, n1\_struct, setam=0.7, min\_nobs=1, /NCLR \\
	        \quad Time   : Fast

	  \item Checkout 'soln\_*' in the Photo directory for the final 
		photometric solutions.  The output given is the filter
		for the solution, and the color, followed by the ZP, AM, CLR fitted
		parameters
		(with corresponding $1\sigma$ errors on the last line).  The final magnitudes are given by:\\
	
	 	\begin{center}
		Mag = m + 25.0 + ZP - AM*obs\_AM - CLR*obs\_clr \\
		\end{center}

		Of course, the +25.0 is simply the default when you use x\_aper or daophot to do standard aperture photometry.  i.e. If you look at the mag\_****.dat files which were created by xdimg\_stdmag, which are stored in the Photo directory, these values are already (m + 25.0) in the above equation.  Note: the quoted zero point applies to the images processed by xdimg\_proc, that is images which already have had the gain applied!

	\end{enumerate}

{\Large  \item Coadd the Images} \\

	\begin{enumerate}

	  \item {\bf xdimg\_proc} :: Process the data images.  Make sure to set
		the correct flat root.  Use the /IR flag if dealing with infrared data, and also provide the name of a bad pixel mask if one exists.  If you want to later coadd the final images using Josh's code, include a FILEOUT name to be written with a list of final files. \\
		\quad OPT Example: IMS = where(n1\_struct.type EQ 'OBJ' AND n1\_struct.flg\_anly NE 0) \\
         	\quad\quad\quad {\bf xdimg\_proc}, n1\_struct, IMS, 'flatroot' \\
                \quad IR Example: IMS =  where(n1\_struct.type EQ 'OBJ' AND n1\_struct.flg\_anly NE 0 and n1\_struct.ccd eq 'NIRC2W') \\
               \quad\quad\quad {\bf xdimg\_proc}, n1\_struct, IMS, 'Flats/SkyFlt\_w\_N', /IR, BPM = 'PixMask/BadPixMask\_w.fits', FILEOUT = 'stackFile' \\
	        \quad Time   : Approx 20 sec / image  

	  \item {\bf x\_offset} :: This routine takes a series of images (in a list)
		and calculates the integer offset between number 2-n and number 1.
		It uses sExtractor to identify objects and then cross-correlates
		the images in a very simple way.
		Copy the default.* files from $\$$XIDL/IMG/Notes/files to a 
		new directory (e.g. '[obj]/offsets') and create a file lists the
		final images. The default output is a file 'offsets' with
		integer offsets.
         	\quad Example: {\bf x\_offset} 'obj.list' INSTR='SITe3'\\
	        \quad Time   : 1min/image \\

	  \item Combine! :: {\bf THIS IS WRONG!!!} Use IRAF's {\it combine} to register the processed
		images.  Be sure to weight by exposure and scale by the median.
		This has the same effect as just summing each image straight-up
		(i.e.\ preserving total counts).
                {\bf After a long discussion with Hsiao-Wen, we determined that what you should really do is subtract the median from each each, and then coadd the median-subtracted images by weighting by the inverse variance and scaling by the exposure time.}

	\end{enumerate}



{\Large  \item Measuring Magnitudes} \\

	We will use sextractor and in order for it to work properly
	the images must be in counts per second (i.e. counts for which Poisson Statistics applies),
	have the same dimensions and be registered [these last requirements
	may no longer be true, I can edit the in-house code to handle
	different sizes and registrations].
	We will use the R band to select our images
	and set the aperture sizes.  We then will use an in-house 
	program to do the photometry
	
	\begin{enumerate}

	  \item Normalizing :: Divide each image by the value in TTIME in
		the header. \\
         	\quad Example: imarith Q0951B / 600.0  Q0951NB \\
	        \quad Time   : Quick 

	  \item Registering :: Either use the Gawiser method where you give the
		offset program the final B and R images and calculate the
		offset and repeat for I or just use your eyeball and ximtool.  
		Then use imcopy to copy subsections of each image into a 
		final 3 images which have the same dimension and are 
		registered. \\
         	\quad Example: imcopy Q0951NB[71:1839,86:2269] Q0951XB \\
         	\quad Example: imcopy Q0951NR[1:1769,1:2184] Q0951XR \\
         	\quad Example: imcopy Q0951NI[2:1770,3:2186] Q0951XI 


	  \item Run Sextractor :: Copy all of the default.* files 
	  	as well as the example .sex files from
		/home/xavier/Keck/LRIS/DLAImg/Sex/ into a subdirectory (Sex).
		Copy the *.sex files to the appropriate name.  Edit the *.sex
		files changing CATALOG\_NAME, MAG\_ZEROPOINT (this is the
		photometric zero point taking airmass into account), 
		SATUR\_LEVEL (55000 / TTIME), 
		SEEING\_FWHM, and CHECKIMAGE\_NAME. Finally, run sextractor
		on the frame for which you wish to identify objects and
		apertures.  We have been using R.\\
         	\quad Example: sex ../Q0132XR.fits -c Q0132+13R\_ap.sex \\

	  \item Run SBlimit :: Check the Surface brightness limit.  Create
		a file SBlimit.inp in Input/
		with (1) Npix in aperture e.g.\ $1''$,
		(2) Nfilt, (3-?) $b1, b2, XB$; $\sigma^{DN}$ repeat \\
         	\quad Ex: SBlimit

	  \item Run lris\_clrphot :: This program will parse the sextractor
		output and calculate mangitudes for BRI based on the output
		from sextractor.  The key is that the apertures are of a fixed
		size for all of the images.  Copy the input file 
		lris\_clrphot.inp from DLAImg/Input into your Input directory
		and edit everything.  The first 9 lines are the fits file,
		total exposure time (s), and Airmass for each filter.
		The next two are the aperture and catalog files from 
		sextractor.
		The last 3 lines are: (1) the number of dummy comments in the
		sextractor output, don't touch this; (2) the photometric
		calibration ($b1, \sigma(b1), b2, \sigma(b2), b3, \sigma(b3)$)
		and (3) the $1\sigma$ limiting magnitude which we will
		use in the case when nothing is detected.  The output is
		fort.11 which has: (i) running number, (ii) x\_pix, 
		(iii) y\_pix; (iv) Bmag, (v) Rmag, (vi) Imag, (vii) sex mag. \\
         	\quad Example: lris\_clrphot \\

	   \item Run lris\_3comp ::  In theory you are ready to rock.  Just
		copy over the lris\_3comp.inp file and get to it. \\
		\quad Example: lris\_3comp \\

	\end{enumerate}
	

\end{enumerate}

\end{document}
